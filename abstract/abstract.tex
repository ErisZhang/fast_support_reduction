\documentclass[12pt]{article}
\usepackage{amsmath}
\newcommand{\calM}{\mathcal{M}}
\newcommand{\calV}{\mathcal{V}}
\newcommand{\calF}{\mathcal{F}}
\newcommand{\calH}{\mathcal{H}}

\newcommand{\bL}{\mathbf{L}}
\newcommand{\bV}{\mathbf{V}}
\newcommand{\bF}{\mathbf{F}}
\newcommand{\bR}{\mathbf{R}}
\newcommand{\bW}{\mathbf{W}}
\newcommand{\bM}{\mathbf{M}}
\newcommand{\bT}{\mathbf{T}}
\newcommand{\bv}{\mathbf{v}}
\newcommand{\bh}{\mathbf{h}}
\newcommand{\bK}{\mathbf{K}}
\newcommand{\bn}{\mathbf{n}}
\newcommand{\bd}{\mathbf{d}}



\begin{document}


\subsection*{Problem formulation}

Let $\calM = (\bV, \bF)$ be mesh living in dimension $d\in \{2,3\}$. Let $\bV = \{\bv_1^T, \cdots, \bv_n^T\}^T \in \R^{n \times d}$ be rest-pose vertex positions. Let $\calH = \{ \bh_1, \cdots, \bh_m \}$ be a set of control handles. Let $\bT_j \in \R^{d \times (d+1)}$ be affine transformation for each handle $\bh_j$. Let $\bT = \{ \bT_1^T, \cdots, \bT_m^T \}^T \in \R^{(d+1)m \times d}$. Linear blend skinning is a deformation method whereby deformed vertices $\bV' = \{ \bv_1'^T, \cdots, \bv_n'^T \}^T \in \R^{n\times d}$ is a weighted linear combination of handles' transformations,
\[
    v_i' = \sum_{j=1}^m w_j(\bv_i) \bT_j 
    \begin{pmatrix} 
    \bv_i \\
    1
    \end{pmatrix} 
\]
where $w_j: \calM \rightarrow \R$ is computed using bounded biharmonic weights. Equivalently
\[
    \bV' = \bM \bT
\]
where,  $\bM \in \R^{n \times (d+1)m}$ is a matrix combining $\bV$ and $\bW$. 

\subsection*{arap energy}

We can define \textit{as-rigid-as-possible} deformation energy that measures local distortion,
\[
    E_{arap}(\bV', \R) = \frac{1}{2} \sum_{f\in \bF} \sum_{(i,j)\in f} c_{ij} || (\bv_i' - \bv_j') - (\bv_i - \bv_j) ||^2
\]
and in matrix form
\[
    E_{arap}(\bV', \R) 
    = tr(\frac{1}{2}\bV'^T \bL \bV' + \bV'^T \bK \bR )
    = tr(\frac{1}{2}\bT^T \tilde{\bL} \bT + \bT^T \tilde{\bK} \bR)
\]
where $\tilde{\bL} = \bM^T \bL \bM \in \R^{(d+1)m \times (d+1)m}$, $\tilde{\bK} = \bM^T \bK \in \R^{(d+1)m \times dn}$ and $\bK$ as defined in the deformation assignment

\subsection*{overhang energy}

An overhanging region that can be 3D printed without support is called \textit{self-supported}. We call the angle between region's tangent plane and printing direction be \textit{self-supported angle} $\alpha$. Let $\alpha_{max}$ be maximum supporting angle. Let $\tau = sin(\alpha_{max})$ be \textit{maximal supporting coefficient}. Let $\partial \calM \subset \bF$ be boundary of mesh $\calM$, i.e. surface faces. A surface face $f$ is \textit{risky} and thus requires support if $\bn(f) \cdot \bd_{p} < -\tau$, where $\bn(f)$ is unit normal of $f$ and $\bd_{p}$ is the printing direction. We can define an overhang energy that measures support required for 3D printing,
\[
    E_{support}(\bV') = \sum_{f\in \partial \calM} min(\bn(f) \cdot \bd_{p} + \tau, 0)
\]

\subsection*{self-intersection energy}

To prevent ARAP from deforming the mesh in such a way as to cause self-intersections, we add a third energy to measure them. We define a \textit{self-intersecting region} as the region inside the mesh that is bordered by intersecting tetrahedra (we assume the input mesh is well-formed and does not contain any self-intersections on its surface). Given $j$ such regions we define the self-intersection energy to be
\[
	E_{intersect}(\bV') = \sum_{i=1}^j V_i
\]

where $V_i$ is the volume of the $i^{\text{th}}$ region. To find these volumes, we trace a ray up from every pixel which can see part of the mesh directly above it and keep track of the number of times it has entered the mesh. "Entering" simply means the direction of the ray is opposite to the vertical component of the normal direction of the surface at that point. If it enters the mesh twice in a row, we know the ray must be in a self-intersecting region. When it exits (i.e. when its direction is the \textit{same} as the vertical component of the normal), we record the length of the ray between the entrance and the exit. The sum of all these lengths produces the combined volume.
\end{document}