
\subsection*{Introduction}

Fused filament fabrication is a popular 3D fabrication process whereby the molten materials are deposited layer by layer. Supporting structures are fabricated concurrently and act as a fixture to support the weights of material that constitute the overhanging regions. Supporting structures introduce material waste and prolong the time required for fabrication. Additionally, they often require manual removal and could potentially introduce damage to the model itself. It is desirable to print models with minimal supporting structures.

There are several approaches to the problem.~\cite{jiang_xu_stringer_2018} One approach focuses on finding the best printing orientation and designing better support generation algorithms.~\cite{vanek_galicia_benes_2014} Alternatively, people have incorporated support structure constraints to topology optimization during model design.~\cite{langelaar_2016} The most relevant approach to this abstract relies on deforming the shape itself for self-support in situation where the geometry of the shape is not critical. There has been attempt to reduce the number of overhanging regions iteratively by deforming an enclosing coarser volumetric mesh.~\cite{hu_jin_wang_2015}.

Linear blend skinning is a fast deformation method. Recent work shown impressive result in automatically generating weights for smooth and intuitive deformation~\cite{jacobson_bounded_biharmonic_weights_2011} and in efficiently computing as-rigid-as-possible deformations by searching in the subspace of skinning deformations.~\cite{jacobson_fast_2012}.

This abstract presents a method for finding natural deformations with reduced support structures. We pose the problem of support reduction as a global optimization problem over the space of linear blend skinning deformations minimizing several objectives, namely
\begin{enumerate}
    \item a local-rigidity measure.~\cite{sorkine_arap_2007}
    \item the volume of support materials.
    \item artifacts, such as self-intersection.
\end{enumerate}
We implemented a prototype and fabricated some model for visualization.

\subsection*{Method}

Let $\calM = (\bV, \bF)$ be rest-pose mesh living in dimension $d\in\{2,3\}$. Let $\bV = \{\bv_1^T, \cdots, \bv_n^T\}^T \in \R^{n \times d}$ be the rest-pose vertex positions. Let $\calM' = (\bV', \bF)$ be deformed mesh.

\subsubsection*{Linear Blend Skinning}

Given a set of handles $\calH = \{ \bh_1, \cdots, \bh_m \}$, we can apply an affine transformation $\bT_j \in \R^{d \times (d+1)}$ on each handle $\bh_j$. Let $\bT = \{ \bT_1^T, \cdots, \bT_m^T \}^T \in \R^{(d+1)m \times d}$. Linear blend skinning is a deformation method whereby the vertex positions on the deformed shape are represented as a weighted linear combination of the handles' transformations,
\[
    \bv_i' = \sum_{j=1}^m w_j(\bv_i) \bT_j 
    \begin{pmatrix} 
    \bv_i \\
    1 
    \end{pmatrix} 
\]
where $w_j: \calM \rightarrow \R$ is computed using bounded biharmonic weights.\cite{jacobson_bounded_biharmonic_weights_2011} Equivalently, $\bV' = \bM \bT$, where $\bM \in \R^{n \times (d+1)m}$ is a matrix combining $\bV$ and $\bW$ as noted in \cite{jacobson_fast_2012}. 

% Note that the energies below are a function of $\bV'$, and hence a function of $\bT$ if we substitute $\bM \bT$ in its place, i.e. 
% \[
%     E_{arap}(\bT) = E_{arap}(\bT, \bM) = E_{arap}(\bV')
% \]
% where $\bM$ is fixed.

% \subsubsection*{Energy Formulation}

% We formulate objectives as stated prevously as functions of deformed vertex positions. Let $E,E_{arap},E_{overhang},E_{intersect} : \R^{n\times d} \to \R$ where $E$ is a weighted sum of 3 energy functions
% \[
%     E = \omega_1 E_{arap} + \omega_2 E_{overhang} + \omega_3 E_{intersect}
% \]


\subsubsection*{As-rigid-as-possible Energy}

To obtain natural deformations, we minimize a distortion energy $E_{arap}:  \R^{n \times d} \to \R$ which measures local deviation from rigidity and acts to preserve shape detail.~\cite{sorkine_arap_2007}
\[
    E_{arap}(\bV') = \frac{1}{2} \sum_{f\in \bF} \sum_{(i,j)\in f} c_{ijf} || (\bv_i' - \bv_j') - \bR_f(\bv_i - \bv_j) ||^2
\]
where $\R_f \in SO(d)$ are per-face rotations and $c_{ijk}\in \R$ are cotangent weights. Let $\R = \{\R_1^T, \cdots, \R_f^T, \cdots, \R_n^T \}^T \in \R^{dn \times d}$. We can find $\bR$ in local step before computing $E_{arap}$,
\[
    \bR = \argmin_{\bR} tr(\bR\tilde{\bK}\bT)
\]
as specified in \cite{jacobson_fast_2012} vis singular value decomposition.

% We can define the \textit{as-rigid-as-possible} deformation energy, which measures local distortion, to be
% \[
%     E_{arap}(\bV', \bR) = \frac{1}{2} \sum_{f\in \bF} \sum_{(i,j)\in f} c_{ij} || (\bv_i' - \bv_j') - \bR(\bv_i - \bv_j) ||^2
% \]
% Reformulating the ARAP energy to be a function of $\bV'$ only, we get
% \[
%     E_{arap}(\bV')
%     \qquad
%     \bR = \argmin_{\bR} tr(\bR\tilde{\bK}\bT)
% \]
% In matrix form, this becomes
% \begin{align*}
% E_{arap}(\bV') 
%     &= tr(\frac{1}{2}\bV'^T \bL \bV' + \bV'^T \bK \bR ) \\
%     &= tr(\frac{1}{2}\bT^T \tilde{\bL} \bT + \bT^T \tilde{\bK} \bR)
% \end{align*}
    
% where $\tilde{\bL} = \bM^T \bL \bM \in \R^{(d+1)m \times (d+1)m}$, $\tilde{\bK} = \bM^T \bK \in \R^{(d+1)m \times dn}$ and $\bK$ as defined in the deformation assignment

\subsubsection*{Overhang Energy}

An overhanging region that can be 3D printed without support is called \textit{self-supported}. We call the angle between the region's tangent plane and printing direction the \textit{self-supported angle} $\alpha$. Let $\alpha_{max}$ be the maximum supporting angle. Let $\tau = \sin(\alpha_{max})$ be the \textit{maximal supporting coefficient}. A surface $f\in \bF$ is considered \textit{risky} and thus requires support if,
\[
    \arccos(\bn_f \cdot \bd_{p}) > \pi + \alpha_{max}
    \quad \rightarrow \quad
    \bn_f \cdot \bd_{p} < - \tau
\]
where $\bn_f$ is unit normal of face $f$ and $\bd_{p}$ is the printing direction. Let $\partial \calM' \subset \bF$ be surface faces of the deformed mesh. Let $A(\cdot)$ be the area function and $c(\cdot)$ be the centroid function for a face $f$. We can approximate the volume of support required for any face risky $f$ by computing the volume of a rectangular prism,
\[
    \lambda(f) = 
    \begin{cases}
        A_{base} h = A_f |\bn_f \cdot \bd_{p}| (\bc_f \cdot \bd_{p}) & \text{if } \bn(f) \cdot \bd_{p} < -\tau \\
        0 & otherwise \\
    \end{cases}
\]
where $A_f$ is the area of the face, $\bc_f$ is the centroid of the face. We define an overhang energy $E_{overhang}: \R^{n\times d} \to \R$ which measures the volume of support required,
\[
    E_{overhang}(\bV') = \sum_{f\in \partial \calM'} \lambda(f)
\]

\subsubsection*{Self-intersection Energy}

One dominant artifact in skinning deformation is self-intersection, which renders the shape not 3D printable. We call a subset of mesh $\calM$ bordered by intersecting tetrahedra the \textit{self-intersecting region}. To quantify the total volume $V_i$ of each self-intersecting region, we define a $k\times k$ grid below the mesh and trace rays up in direction of the printing direction from every grid point. A ray is inside a self-intersecting region if $\bd_{p} \cdot \bn_f < 0$ for 2 consecutively incident faces. Summation of distances that rays traversed inside a self-intersecting region approximates the volume. We can define a self-intersection energy $E_{intersect}: \R^{n\times d} \to \R$ which measures the total volume of the self-intersecting regions as follows,
\[
    E_{intersect}(\bV') = \sum_i V_i 
\]

% traversed inside each region 




% To minimize self-intersection, we define a \textit{self-intersection energy}

% To prevent ARAP from deforming the mesh in such a way as to cause self-intersections, we add a third energy to measure them. We define a \textit{self-intersecting region} as the region inside the mesh that is bordered by intersecting tetrahedra. Given $j$ such regions we define the self-intersection energy to be
% \[
% 	E_{intersect}(\bV') = \sum_{i=1}^j V_i
% \]

% where $V_i$ is the volume of the $i^{\text{th}}$ region. To find these volumes, we define a grid below the mesh and trace a ray up from every grid point. If it enters the mesh twice in a row, we know the ray must be in a self-intersecting region. "Entering" simply means the direction of the ray is opposite to the vertical component of the normal direction of the surface at that point. When it exits this region, we record the distance the ray traversed inside it. The sum of all these lengths produces an approximation of the combined volume. The finer the grid, the more accurate the approximation.


\subsection*{Optimization}

We define an objective function $E(\bx): \R^n \rightarrow \R$ as the sum of the three energy terms $E = E_{arap} + E_{overhang} + E_{intersect}$
and minimize $E$ using the particle swarm optimization. Let $\bx \in \R^{2dm}$ consisting a representation of centroid and rotation (as Euler's angle) about the centroid of for each edge handle. For shape in 

Finally, we can convert each component to $\bT_j$, affine transformations for edge handles, from which can compute deformed shaped $\bV' = \bM \bT$ with LBS.

\subsection*{Results?}

\subsection*{Future work}
In future work, we would like to speed up the calculations with a GPU. This would reduce wait time for the user and introduce the possibility of user interactivity, so that a user could have some control over the model's final position. We would also like to have the optimizer consider how the deformation shifts the model's center of mass, so that the model would still be able to stand on its own once it has been printed.